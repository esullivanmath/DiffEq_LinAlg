\chapter{Partial Fractions}
In this appendix we explore the algebraic notion of partial fractions.  The idea is
simple:\\ How do we undo the addition of fractions?

Let's first consider some elementary arithmetic.
\begin{flalign*}
    \frac{1}{3} + \frac{2}{5} = ?
\end{flalign*}
To add the two fractions you need common denominators, in this case $15$.  We rewrite the
fractions as 
\[ \frac{1}{3} + \frac{2}{5} = \frac{5}{15} + \frac{6}{15} \]
and now that we're comparing like parts we add the numerators to get
\[ \frac{1}{3} + \frac{2}{5} = \frac{5}{15} + \frac{6}{15} = \frac{11}{15}. \]

What if we wanted to go the other way?  That is, what if we have the fraction $11/15$ and
we wanted to know where it came from.  If we consider the prime factorization of the
denominator and conjecture that the fractions can be split up with these factors as the
denominators of separate fractions then the problem becomes
\[ \frac{11}{15} = \frac{11}{3 \cdot 5} = \frac{A}{3} + \frac{B}{5} \]
where $A$ and $B$ are just numbers that we need to find.  Obviously there are many different answers to this
inverse questions (since we can get infinitely many equivalent fractions).  If we multiply
both sides of this new equation by $15$ we get
\[ 11 = 5A + 3B \]
and for each choice of one variable we get another.  In particular, if we choose $A = 1$
then simple algebra tells us that $B=2$ and we have successfully split the fraction
$11/15$ into the sum of $1/3$ and $2/5$.  


% As you likely saw in the previous problem there is a handy algebraic technique that will
% allow you to do the integrations required from separation of variables.  In these notes I
% will give you a few examples of partial fractions.  It is up to the reader to seek outside
% resources if these examples do not suffice.

Now let's consider the algebraic problem of taking a fraction and splitting it into a sum
of fractions.  The notion is still the same: conjecture that the factors of the
denominator are the denominators of the separate fractions and then do some detective work
to find the numerators.  In the remainder of this appendix we'll give several examples of
this idea.  We leave it up to the reader to actually find the common denominators and do
the algebra to verify that indeed the right-hand side from each example is equal to the
left-hand side.

\begin{example}
    Use partial fractions to write $\frac{4}{x(x-3)}$ as a sum or difference of two
    fractions. \\
    {\bf Solution:} We start by writing the fraction as
    \[ \frac{4}{x(x-3)} = \frac{A}{x} + \frac{B}{x-3}. \]
    This choice is made since if we were to find the common denominator of the right-hand
    side we would have the desired denominator on the left-hand side.  Next we clear all
    of the fractions by multiplying the common denominator yielding
    \[ 4 = A(x-3) + B(x). \]
    At this point we know that the equal sign must be true for \underline{all} values of
    $x$ so we can choose some convenient values to tease out $A$ and $B$.
    \begin{itemize}
        \item If $x = 3$ then $x-3=0$ and we get $4 = 3B$ which implies that $B =
            \frac{4}{3}$.
        \item If $x =0$ then we get $4 = -3A$ which implies that $A =
            -\frac{4}{3}$
    \end{itemize}
    Therefore
    \[ \frac{4}{x(x-3)} = -\frac{4}{3x} + \frac{4}{3(x-3)}. \]
\end{example}

\begin{example}
    It can be shown that 
    \[ \frac{6x}{(x-1)(x+1)(x+2)} = \frac{1}{x-1} + \frac{3}{x+1} - \frac{4}{x+2}. \]
    {\bf Partial Justification:} Start by observing that the denominator of the left-hand
    fraction is factored so we split into three fractions with the factors as the
    denominators:
    \[ \frac{6x}{(x-1)(x+1)(x+2)} = \frac{A}{x-1} + \frac{B}{x+1} + \frac{C}{x+2}. \]
    Clearing the fractions gives
    \[ 6x = A(x+1)(x+2) + B(x-1)(x+2) + C(x-1)(x+1). \]
    Now consider convenient choices of $x$
    \begin{itemize}
        \item If $x=-1$ then:
            \[ -6 = A(0)(1) + B(-2)(1) + C(-2)(0) \quad \implies \quad B = 3. \]
        \item If $x=1$ then:
            \[ 6 = A(2)(3) + B(0)(3) + C(0)(2) \quad \implies \quad A = 1. \]
        \item If $x=-2$ then:
            \[ -12 = A(-1)(0) + B(-3)(0) + C(-3)(-1) \quad \implies \quad C = -4. \]
    \end{itemize}
    Hence
    \[ \frac{6x}{(x-1)(x+1)(x+2)} = \frac{1}{x-1} + \frac{3}{x+1} - \frac{4}{x+2}. \]
\end{example}

\begin{example}
    In this problem we will see repeated linear factors.
    \[ \frac{x^2 + 1}{x(x-1)^3} = \frac{A}{x} + \frac{B}{x-1} + \frac{C}{(x-1)^2} +
    \frac{D}{(x-1)^3} \]
    Notice that the repeated factor gets repeated for all powers.  Let's clear the
    fractions just as before and see what we get
    \[ x^2 + 1 = A(x-1)^3 + Bx(x-1)^2 + Cx(x-1) + Dx. \]
    If we take $x=0$ then $A = -1$.  If we take $x=1$ then $D = 2$.  However, you'll
    notice that these two choices do not allow us to easily find $B$ and $C$ so we expand
    the polynomial on the right-hand side, gather like terms, and match coefficients.
    That is
    \begin{flalign*}
        x^2 + 1 &= A\left( x^3 - 3x^2 + 3x - 1 \right) + B\left( x^3 - 2x^2 + x \right) +
        C\left( x^2 - x \right) + Dx \\
        \implies x^2 + 1 &= (A+B)x^3 + (-3A -2B+C) x^2 + (3A + B - C + D) x - A
    \end{flalign*}
    Matching the coefficients of like terms we get
    \begin{flalign*}
        A+B &= 0 \quad \text{(cubic terms)} \\
        -3A - 2B + C &= 1 \quad \text{(quadratic terms)} \\
        3A + B - C + D &= 0 \quad \text{(linear terms)} \\
        -A &= 1 \quad \text{(constant terms)} \\
    \end{flalign*}
    Since $A = -1$ we must have $B = 1$ and therefore $C = 0$.  Therefore
    \[ \frac{x^2 + 1}{x(x-1)^3} = -\frac{1}{x} + \frac{1}{x-1} + \frac{0}{(x-1)^2} +
    \frac{2}{(x-1)^3} \]
\end{example}

\begin{example}
    In this final example we'll show what happens with an irreducible quadratic.
    \[ \frac{x-3}{x(x^2+3)} = \frac{A}{x} + \frac{Bx+C}{x^2+3}. \]
    Notice that the numerator associated with the irreducible quadratic is a linear
    function with unknown parameters.  Clearing the fractions we get
    \[ x-3 = A(x^2 + 3) + (Bx+C)(x). \]
    If we take $x=0$ then $-3 = 3A$ which implies that $A = -1$.  Expanding both sides of
    the equation and matching like terms gives
    \[ x-3 = (A+B)x^2 + Cx + 3A \]
    which implies that $A+B=0$ and $C=1$.  Therefore $B = 1$ and
    \[ \frac{x-3}{x(x^2+3)} = -\frac{1}{x} + \frac{x+1}{x^2+3}. \]
\end{example}

\begin{technique}[Partial Fractions Decomposition]
    Below are several cases of fractions that require partial fractions along with their
    separated forms.
    \begin{flalign*}
        \frac{px+q}{(x-a)(x-b)} &= \frac{A}{x-a} + \frac{B}{x-b} \quad \text{(for $a\ne
        b$)} \\
        \frac{px+q}{(x-a)^2} &= \frac{A}{x-a} + \frac{B}{(x-a)^2} \\
        \frac{px+q}{(x-a)^3} &= \frac{A}{x-a} + \frac{B}{(x-a)^2} + \frac{C}{(x-a)^3}\\
        \frac{px^2 + qz + r}{(x-a)(x-b)(x-c)} &= \frac{A}{x-a} + \frac{B}{x-b} +
        \frac{C}{x-c} \\
        \frac{px^2 + qz + r}{(x-a)^2(x-b)} &= \frac{A}{x-a} + \frac{B}{(x-a)^2} +
        \frac{C}{x-b} \\
        \frac{px^2 + qx + r}{(x-a)(x^2+bx+c)} &= \frac{A}{x-a} + \frac{Bx+C}{x^2+bx+c}
        \quad \text{(where $x^2 + bx + c$ cannot be factored)}
    \end{flalign*}
\end{technique}

\begin{problem}
    Now let's use partial fractions to solve a separable differential equation.
    Solve the logistic population equation
    \[ \frac{dP}{dt} = 0.2 P \left( 1-\frac{P}{10} \right) \quad \text{with} \quad P(0) =
        1 \]
    Start by separating the variables (leaving the $0.2$ on the right) and then looking up
    the appropriate partial fractions decomposition for splitting up the fraction that
    appears.  After that you'll get to do a whole bunch of algebra \ldots have fun!!
\end{problem}
\solution{
    \begin{flalign*}
        \frac{dP}{dt} &= 0.2 P \left( 1-\frac{P}{10} \right) \\
        \frac{dP}{P(1-P/10)} &= 0.2 dt \\
        \int \frac{dP}{P(1-P/10)} &= \int 0.2 dt \\
        \int \frac{dP}{P(1-P/10)} &= 0.2t + C \\
    \end{flalign*}
    Now we need to split the fraction on the left.
    \[ \frac{1}{P(1-P/10)} = \frac{A}{P} + \frac{B}{1-P/10} \quad \implies \quad 1 =
        A(1-P/10) + BP \]
    If we take $P=0$ then $A=1$.  If we take $P=10$ then $B = 1/10$.  Therefore,
    \[ \frac{1}{P(1-P/10)} = \frac{1}{P} + \frac{1}{10(1-P/10)} \]
    and we can now integrate the left-hand side to get (with a little bit of
    $u$-substitution)
    \[ \int \frac{1}{P(1-P/10)}dP = \int \frac{1}{P} dP + \frac{1}{10} \int
        \frac{1}{1-P/10}dP = \ln(P) - \ln(1-P/10) \]
    Therefore,
    \[ \ln(P)-\ln(1-P/10) = 0.2t+C \implies \ln \left( \frac{P}{1-P/10} \right) = 0.2t + C
        \implies \frac{P}{1-P/10} = Ce^{0.2t} \]
        \[ \implies P = Ce^{0.2t} \left( 1-\frac{P}{10} \right) \implies P + P
            \frac{Ce^{0.2t}}{10} = Ce^{0.2t} \implies P\left( 1+\frac{Ce^{0.2t}}{10}
            \right) = Ce^{0.2t} \]
            \[ \implies \boxed{P(t) = \frac{Ce^{0.2t}}{1+(C/10)e^{0.2t}} } \]
    Assuming now that $P(0) = 1$ we can take any of the algebraically equivalent forms of
    the final answer to tease out the value of $C$.  In particular, 
    \[ \frac{1}{1-1/10} = C e^{0.2(0)} \implies C = \frac{1}{9/10} = \frac{10}{9} \implies
        P(t) = \frac{10e^{0.2t}}{9 + e^{0.2t}}\]
}


